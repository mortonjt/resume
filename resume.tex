\documentclass{resume} % Use the custom resume.cls style
\usepackage[left=0.75in,top=0.6in,right=0.75in,bottom=0.6in]{geometry} % Document margins

\name{James T. Morton}
\address{744 High Meadow Lane \\ Oxford, OH, 45056}
\address{(513)~$\cdot$~907~$\cdot$~9853 \\ jamietmorton@gmail.com} % Your phone number and email

\begin{document}
\begin{rSection}{Education}

  %% {\bf University of California, Berkeley} \hfill {\em June 2004} \\ 

  \begin{tabular}{ll}
    \textbf{Undergraduate}    & Miami University \\
    2010 - Present   & B.S. in Computer Science \\
    & B.S. in Electrical Engineering \\
    & B.S. in Mathematics and Statistics \\
    & B.S. in Engineering Physics \\
    & GPA: 3.87/4.0 \\
  \end{tabular} \\
  \begin{tabular}{ll}
    \textbf{Study Abroad}    &   Hong Kong University of Science and Technology \\
    Spring 2012 \\
  \end{tabular}

  \begin{tabular}{ll}
    \textbf{Secondary}    & Talawanda High School \\
    2006 - 2010           & Class rank: 2/275 \\
    &GPA: 4.45/4.0 \\
  \end{tabular} \\
\end{rSection}
\begin{rSection}{Skills}
  \textbf{Technical Skills}\\[1 mm]
  \begin{tabular}{llll}
     $\bullet$ Python   &  $\bullet$ C/C++         &  $\bullet$ Java   &   $\bullet$ \LaTeX\\
     $\bullet$ Matlab   &  $\bullet$ R             &  $\bullet$ Bash   &   $\bullet$ Unix \\ 
     $\bullet$ ROS      &  $\bullet$ Hadoop        &  $\bullet$ CUDA   &   $\bullet$ git\\
     $\bullet$ Verilog  &  $\bullet$ MIPS assembly &  $\bullet$ MySQL  &   $\bullet$ svn\\
  \end{tabular}\\[1 mm]
  \textbf{Languages}\\[1 mm]
  \begin{tabular}{l}
    $\bullet$ Chinese --  Working Proficency\\
    $\bullet$ English --  Native speaker
  \end{tabular}
\end{rSection}
\begin{rSection}{Experience}
  \textbf{Research Experience}\\[1 mm]
  \textit{Data Scientist Intern} \hfill John Hopkins University, MD , Summer 2013 \\ [1 mm]
  $\bullet$ Worked with Dr. Benjamin Langmead to develop scalable RNAseq Analysis software \\
  $\bullet$ Integrated Hadoop framework for RNAseq analysis software \\
  $\bullet$ Developed a novel spliced alignment algorithm using Bowtie \\
  $\bullet$ Used SWIG to interface Python and C code \\[3 mm]
  \textit{Undergraduate Research Program} \hfill Cold Spring Harbor Laboratories, NY , Summer 2012 \\ [1 mm]
  $\bullet$ Worked with Dr. Thomas Gingeras and Dr. Alex Dobin\\
  $\bullet$ Developed software that maps reads between the reference and personal genome \\
  $\bullet$ Studied Allele Specific Expression in a personal genome \\[3 mm]
  \textit{Research Assistant} \hfill Miami University, OH , Spring 2011 - Fall 2011 \\ [1 mm]
  $\bullet$ Worked with Dr. John Karro and Dr. Chun Liang\\
  $\bullet$ Designed Hidden Markov Model software to identify poly(A) tails in RNAseq data\\
  $\bullet$ Designed Profile Hidden Markov Model software to identify adapter sequences in RNAseq data\\
  $\bullet$ Contributed HMMER parser to Biopython\\[20 mm]
  \textit{Research Assistant} \hfill Miami University, OH , Summer 2010 \\ [1 mm]
  $\bullet$ Worked with Dr. Qihou Zhou on processing incoherent scattering radar data\\
  $\bullet$ Developed signal processing algorithms to extract atmospheric parameters from this data\\[3 mm]
  \textit{Engineering Aide} \hfill Wright Patterson Air Force Base, OH, Summer 2010 \\ [1 mm]
  $\bullet$ Designed and implemented a time difference of arrival localization algorithm \\
  $\bullet$ Programmed USRP using GNU radio for signal transmission and receiving \\[3 mm]
  \textit{Wright Scholar} \hfill Wright Patterson Air Force Base, OH, Summer 2009 \\ [1 mm]
  $\bullet$ Studied cognitive radio, radar, and GPS concepts and techniques \\[3 mm]
  \textbf{Teaching Experience}\\[1 mm]
  \textit{Teaching Assistant} \hfill Miami University, OH, Spring 2011 \\ [1 mm]
  $\bullet$ Assisted Professor Mostafa Modirrousta in teaching of two sections of Intro to Engineering labs\\
  $\bullet$ Graded lab reports for a class of 32 students \\ [3 mm]
  \textit{Teaching Assistant} \hfill Miami University, OH, Spring 2008 \\ [1 mm]
  $\bullet$ Assisted Professor Felice Marcus to teach a class of Chinese engineers English \\[3 mm]
  \textbf{Class Projects}\\[1 mm]
  \textit{Intelligent Ground Vehicle Competition} \hfill Rochester, MI , Summer 2013 \\ [1 mm]
  $\bullet$ Interfaced with the Bumblebee2 Stereo camera \\
  $\bullet$ Developed computer vision algorithms to identify white lines on the ground \\
  $\bullet$ Interfaced CUDA with ROS to speed up computer vision software \\
  $\bullet$ Helped install ROS framework on autonomous vehicle \\[1 mm]
  \textit{Embedded Systems} \hfill Miami University, OH, Spring 2013\\
  $\bullet$ Participated in a group project and built a LED POV sphere\\[1 mm]
  \textit{Artificial Intelligence} \hfill Miami University, OH, Fall 2012\\
  $\bullet$ Participated in a group project to develop an AI player to play Breakthrough\\
  $\bullet$ Developed a C++/Java interface using JNI\\[1 mm]
  \textit{Databases} \hfill Miami University, OH, Fall 2011\\
  $\bullet$ Participated in a group project to develop a social media program for recipe sharing \\
  $\bullet$ Developed a Java Swing application for the user interface \\
  $\bullet$ Constructed a SQL Query to store information about users and recipes\\[1 mm]
  \textit{Digital Systems and Design} \hfill Miami University, OH, Fall 2010\\
  $\bullet$ Participated in a group project for Redhawk Duals, a competitive FPGA-based video game\\
  $\bullet$ Designed and implemented a VGA interface for a FPGA\\
  $\bullet$ Implemented a Finite State Machine to solve the Longest Path Problem\\[1 mm]
\end{rSection}
\begin{rSection}{Other Activities}
  \begin{tabular}{ll}
    $\bullet$ & National Society of Collegiate Scholars, Fall 2012 - Present\\
    $\bullet$ & Miami University Collegiate Chorale, Fall 2012.\\
    $\bullet$ & Institute of Navigation Autonomous Snowplow Competition support team, 2010-Present\\
    $\bullet$ & IEEE Miami Student Chapter Treasurer, Fall 2011- Spring 2012\\
    $\bullet$ & Miami University Men’s Glee Club, Fall 2011-Spring 2012\\
    $\bullet$ & ACM Programming Contest, Fall 2011, Fall 2012\\
    $\bullet$ & International Global Game Jam, Spring 2011\\
    $\bullet$ & Miami University Symphony Orchestra, Spring 2010, Fall 2009\\[5 mm]
  \end{tabular}
\end{rSection}
\begin{rSection}{Honors}
  \begin{tabular}{ll}
    $\bullet$ & Goldwater Scholar, 2013\\
    $\bullet$ & Harrison Scholar, Miami University, 2010-2014\\
    $\bullet$ & Provost Academic Achievement Award, Miami University, 2012\\
    $\bullet$ & Ohio Space Grant Scholar Award, NASA, 2012 - 2013\\
    $\bullet$ & NSF REU, CSHL, Summer 2012\\
    $\bullet$ & Dean’s List, Miami University, 2010-12\\
    $\bullet$ & Mary Jean and Joseph R. Priest Scholarship, Department of Physics, Miami University, 2012\\
    $\bullet$ & President List, Miami University, 2010-11\\
    $\bullet$ & Nestle Scholar, Computer Sci. and Software Eng. Dept, Miami University, 2011\\
    $\bullet$ & Faculty Prize, Department of Mathematics, Miami University, 2011\\
    $\bullet$ & Joseph A. Culler Award, Department of Physics, Miami University, 2011\\
    $\bullet$ & R.L. Edwards Scholarship, Department of Physics, Miami University, 2011\\
    $\bullet$ & Joseph A. Culler Award, Department of Physics, Miami University, 2010\\
    $\bullet$ & NSF Travel Grant, Presenting a poster paper at Coupling, Energetics, \\
              & and Dynamics of Atmospheric Regions workshop in Boulder, CO, 2010\\
    $\bullet$ & Second place team leader, Institute of Navigation Mini-Urban Challenge, Ohio competition, 2010\\
    $\bullet$ & Wright Scholar, Air Force Research Laboratory, Wright Patterson Air Force Base, 2009\\
    $\bullet$ & First place team member, Institute of Navigation Mini-Urban Challenge, Ohio competition, 2009\\
  \end{tabular}
\end{rSection}

\begin{rSection}{Presentations}
  \begin{tabular}{ll}

    1.  & Morton, J., P., Abrudan, J. Karro, C. Liang , “Sequence classification of homopolymer emissions \\
        & (SCOPE),” Great Lakes Bioinformatics Conference, Pittsburgh, PA, 2013\\
    2.  & Morton, J., P., Abrudan, J. Karro, C. Liang , “Sequence classification of homopolymer emissions \\
        & (SCOPE),” Ohio Space Grant Consortium, Cleveland OH, 2013\\
    3.  & Morton, J., P., Abrudan, J. Karro, C. Liang , “Sequence classification of homopolymer emissions \\
        & (SCOPE),” IEEE 2nd International Conference on Computational Advances in Bio and \\
        & Medical Sciences, ICCABS 2012, Las Vegas, NV, February 2012\\
    4.  & Morton, J., J. Karro, C. Liang, “A novel approach for identifying poly(A) tails in \\
        & raw cDNA sequence data using General Hidden Markov Models,” Genome Informatics \\
        & Cold Spring Harbor, NY, November 2011.\\
    5.  & Morton, J., C., Liang, and J. Karro. scrapplusplus -- SCRAP \\
        & Sequence Cleaning and Removal of Adapter Sequences using Profile HMMs\\
        & Google Project Hosting. Retrieved from http://code.google.com/p/scrapplusplus, 2012\\
    6.  & Morton, J., J. Karro, and C. Liang. scopeplusplus -- SCOPE \\
        & Sequence Classification Of homoPolymer Emissions. Google Project Hosting.\\
        & Retrieved from http://code.google.com/p/scopeplusplus, 2012\\
    7.  & Morton, J., “MiniUrban Challenge: An Institute of Navigation Autonomous Robot Competition,”\\
        & Miami University, November 2010 \\
    8.  & Santana, J., J. Morton, Q. Zhou, “A Fuzzy Logic Approach to Extract Plasma Line Frequencies\\
        & from Arecibo Incoherent Scatter Radar Measurements,” Coupling, Energetics, and \\
        & Dynamics of Atmospheric Regions Workshop, Boulder, CO, June 2010.\\
    9.  & Morton, J., C. Meikle, K. Danielson, “Dumbo: An Intelligent Lego Robot,”\\
        & Mini-Urban Challenge, Dayton, OH,June 2010.\\
    10.  & Brezheva, D., J. Morton, C. Meikle, K. Danielson, S. Joseph, R. Morton,\\
        & “StarCruizer: An Intelligent Lego Robot,”Mini-Urban Challenge, Dayton, OH, June 2009\\
  \end{tabular}
\end{rSection}

\end{document}


