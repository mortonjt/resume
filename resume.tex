\documentclass{resume} % Use the custom resume.cls style
\usepackage[left=0.75in,top=0.6in,right=0.75in,bottom=0.6in]{geometry} % Document margins
\usepackage{natbib}
\usepackage{bibentry}
\nobibliography{ref.bib}
\bibliographystyle{apalike}
\name{James (Jamie) T. Morton}
\address{(513)~$\cdot$~907~$\cdot$~9853 \\ jamietmorton@gmail.com \\ https://github.com/mortonjt}% Your phone number and email and github
\begin{document}
\begin{rSection}{Education}
  \begin{tabular}{ll}
    \textbf{Graduate}    & University of California, San Diego  \\
    2015 - Present  & PhD student in Computer Science \\
    \textbf{Graduate}    & University of Colorado, Boulder \\
    2014 - 2015  & PhD student in Computer Science \\
                    & Integrative Quantitative Biology Program \\
    \textbf{Undergraduate}    & Miami University \\
    2010 - 2014   & Four B.S. Degrees with majors in\\
    & Computer Science (Cum Laude)\\
    & Engineering (Cum Laude)\\
    & Mathematics and Statistics \\
    & Engineering Physics \\
    \textbf{Study Abroad} &  Hong Kong University of Science and Technology   \\
    Spring 2012 \\
  \end{tabular}
\end{rSection}
\begin{rSection}{Honors}
  \begin{tabular}{ll}
    $\bullet$ & \textbf{NSF Graduate Fellow}, 2015 - 2018 (Started date deferred from Fall 2014 as requested)\\
    $\bullet$ & \textbf{Integrated Quantitative Biology Fellowship}, University of Colorado Boulder, 2014 - 2015\\
    $\bullet$ & \textbf{National Barry Goldwater Scholar}, 2013\\
    $\bullet$ & \textbf{Benjamin Harrison Scholar}, Miami University, 2010-2014\\
    $\bullet$ & \textbf{First place}, Institute of Navigation (ION) Autonomous Snowplow Competition, 2014\\
    $\bullet$ & \textbf{NSF REU}, Cold Spring Harbor Laboratories, Summer 2012\\
    $\bullet$ & \textbf{Provost Academic Achievement Award}, Miami University, 2012\\
    $\bullet$ & \textbf{Ohio Space Grant Scholar Award}, NASA, 2012 - 2014\\
    $\bullet$ & \textbf{Dean's List}, Miami University, 2010-13\\
    $\bullet$ & \textbf{R.L. Edwards Scholarship}, Department of Physics, Miami University, 2011, 2013\\
    $\bullet$ & \textbf{Mary Jeannette and Clifford Harvey Scholarship}, Department of Mathetmatics, Miami U., 2013\\
    $\bullet$ & \textbf{Mary Jean and Joseph R. Priest Scholarship}, Department of Physics, Miami University, 2012\\
    $\bullet$ & \textbf{President List}, Miami University, 2010-11\\
    $\bullet$ & \textbf{Nestle Scholar}, Computer Sci. and Software Eng. Dept, Miami University, 2011\\
    $\bullet$ & \textbf{Faculty Prize}, Department of Mathematics, Miami University, 2011\\
    $\bullet$ & \textbf{Joseph A. Culler} Award, Department of Physics, Miami University, 2010,2011\\
    $\bullet$ & \textbf{NSF Travel Grant}, Coupling, Energetics, \& Dynamics of Atmospheric Regions workshop, 2010\\
    $\bullet$ & \textbf{Wright Scholar}, Air Force Research Laboratory, Wright Patterson Air Force Base, 2009\\
  \end{tabular}
\end{rSection}
\begin{rSection}{Research Interests}
  Microbial ecology, multi-omics data fusion, functional genomics, high dimensional statistics, compositional data analysis,
  machine learning.
\end{rSection}
\begin{rSection}{Publications}
  \begin{enumerate}
    \item \bibentry {morton2017balance}
    \item \bibentry {morton2017uncovering}
    \item \bibentry {amir2017correcting}
    \item \bibentry {amir2017deblur}
    \item \bibentry {vazquez2017bringing}
    \item \bibentry {vrbanac2017elegan}
    \item \bibentry {hill2017parkinson}
    \item \bibentry {hemmings2017microbiome}
    \item \bibentry {reber2016immunization}
    \item \bibentry {gilbert2016microbiome}
    \item \bibentry {nellore2016rail}
    \item \bibentry {petras2016mass}
    \item \bibentry {barberan2015ecology}
    \item \bibentry {morton2015large}
    \item \bibentry {morton2014scope++}
  \end{enumerate}
\end{rSection}
\begin{rSection}{Presentations}
  \begin{itemize}
    \setlength\itemsep{0em}
    \item  Morton et al. Balances Reveal Microbial Niche Differentiation. CODAwork (2017)
    \item  Morton et al. From Probabilities to Balances: An Alternative Approach. Information Theory and Applications Workshop (2016)
    \item  Morton et al. From Probabilities to Balances: An Alternative Approach Random Processes and Time Series: Theory and Applications (2016)
    \item  Reber et al. An immunization strategy for prevention of post-traumatic stress disorder (PTSD) promotes stress resilience in mice.
                 University California San Diego Pediatrics Symposium  (2016)
    \item  Reber et al. Immunization with a heat-killed preparation of the environmental bacterium Mycobacterium vaccae
                promotes stress resilience in mice. DNA Day (2015)
    \item   Morton, J., Lladser M., Knight R., Uncovering the Unknown: A New Approach in
                Analyzing Microbiome Data NSF Data Science Workshop, 2015
    \item   Morton, J., Freed, S. Lee, S. Friedberg, I. Prediction of Bacteriocin Associated Operons
                Rocky Mountain Bioinformatics Conference, 2014
    \item   Morton, J., Freed, S. Lee, S. Friedberg, I. A pipeline for Identifying Bacteriocin-Associated
                Gene Clusters. ISMB Boston, 2014
    \item   Morton, J., Freed, S. Lee, S. Friedberg, I. Discovering the Next Antibiotic
                Ohio Space Grant Consortium, Cleveland OH, 2014
    \item   Morton, J., P., Abrudan, J. Karro, C. Liang , Sequence classification of homopolymer emissions
                (SCOPE), Great Lakes Bioinformatics Conference, Pittsburgh, PA, 2013
    \item   Morton, J., P., Abrudan, J. Karro, C. Liang , Sequence classification of homopolymer emissions
                (SCOPE), Ohio Space Grant Consortium, Cleveland OH, 2013
    \item   Morton, J., P., Abrudan, J. Karro, C. Liang , Sequence classification of homopolymer emissions
                (SCOPE), IEEE 2nd International Conference on Computational Advances in Bio and
                Medical Sciences, ICCABS 2012, Las Vegas, NV, February 2012
    \item   Morton, J., J. Karro, C. Liang, A novel approach for identifying poly(A) tails in
                raw cDNA sequence data using General Hidden Markov Models, Genome Informatics
                Cold Spring Harbor, NY, November 2011.
  \end{itemize}
\end{rSection}

\begin{rSection}{Professional Memberships and Services}
  \begin{tabular}{ll}
    $\bullet$ & Poster Reviewer for ISMB 2015  2014-2016\\
    $\bullet$ & International Society of Computational Biology Student member, Summer  2014-Present\\
    $\bullet$ & Sigma Pi Sigma,Tau Beta Pi,Eta Kappa Nu Spring 2014-2014 \\
    $\bullet$ & National Society of Collegiate Scholars, Fall 2012 - Spring 2013\\
    $\bullet$ & Association for Computing Machinery Student member, Fall  2011-2014 \\
    $\bullet$ & Institute of Electrical and Electronics Engineers Student member, Fall  2011-2016 \\
    $\bullet$ & IEEE Miami Student Chapter Treasurer, Fall 2011- Spring 2012\\
  \end{tabular}
\end{rSection}

\begin{rSection}{Workshops}
  \begin{tabular}{ll}
    $\bullet$ & Instructor. Qiime2 workshop. UBC Vancouver August 23-25, 2017 \\
    $\bullet$ & Instructor. Qiime2 workshop. UBC Kelowna August 21-22, 2017 \\
    $\bullet$ & Teaching Assistant. Qiime2 workshop.  Las Vegas June 21-23, 2017 \\
    $\bullet$ & Teaching Assistant. STAMPS
    Woodshole, MA, August 2-13 2016 \\
  \end{tabular}
\end{rSection}

\begin{rSection}{Student Mentoring}
  During my studies as a PhD students, I have mentored and co-authored with the following students.
  \begin{tabular}{ll}
    $\bullet$  &   Jue Wang (Summer 2017) \\
    $\bullet$  &   Liam Toran (Summer 2016)\\
    $\bullet$  &   Kayla Orlinsky (Spring 2016) \\
  \end{tabular}
\end{rSection}

\begin{rSection}{Skills}
  \textbf{Foreign Language Skills}\\[1 mm]
  \begin{tabular}{l}
    $\bullet$ Chinese --  Working Proficency in Mandarin and written Chinese\\
  \end{tabular}\\
  \textbf{Technical Skills}\\[1 mm]
  \begin{tabular}{lllllll}
     $\bullet$ Python   &  $\bullet$ C/C++ &  $\bullet$ Java  & $\bullet$ Javascript &  $\bullet$ \LaTeX   &  $\bullet$ ROS    &  $\bullet$ Hadoop\\
     $\bullet$ Matlab   &  $\bullet$ R     &  $\bullet$ Unix   & $\bullet$ SQL      &  $\bullet$ OpenCL   &  $\bullet$ CUDA   &   $\bullet$ git\\
  \end{tabular}\\[1 mm]
\end{rSection}
\begin{rSection}{Open Source Contributions}
  \begin{tabular}{ll}
      $\bullet$ & Gneiss (Core Maintainer)\\
      $\bullet$ & Sci-kit Bio (Developer)\\
      $\bullet$ & Emperor (Developer)\\
      $\bullet$ & Qiime2 (Contributor)\\
      $\bullet$ & Deblur (Contributor)\\
      $\bullet$ & Micronota (Developer)\\
      $\bullet$ & BOA: Bacteriocin Operon Associator (Lead Developer)\\
      $\bullet$ & SCOPE++: Sequence Classification Of homoPolymer Emissions (Lead Developer)\\
      $\bullet$ & Rail-RNA (Contributor)\\
      $\bullet$ & Scipy (Contributor)\\
      $\bullet$ & Biopython (Contributor)\\
  \end{tabular}
\end{rSection}

\nobibliography{ref}
\end{document}
